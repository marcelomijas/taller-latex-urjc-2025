% Esta plantilla ha sido creada por Marcelo Marcelo Moreno Porras,
% como parte del taller de Creación de Hojas de Referencia con LaTex
% de las Cuartas Jornadas de Cultura Libre de la Universidad Rey Juan Carlos
% CC-BY-4.0 license

\documentclass{article}

% ----- document -----
\begin{document}

\section{Ejemplos básicos}
Este es un ejemplo de un documento muy básico con \LaTeX. Aquí se muestra cómo utilizar \textbf{texto en negrita} y \textit{texto en cursiva}.

\subsection{Ecuaciones}
La fórmula cuadrática es: 
\begin{equation}
x = \frac{-b \pm \sqrt{b^2-4ac}}{2a}
\end{equation}

\subsection{Listas}
Ejemplo de una lista:
\begin{itemize}
    \item Primer elemento de la lista
    \item Segundo elemento de la lista
    \begin{itemize}
        \item Item anidado
    \end{itemize}
    \item Tercer elemento de la lista
\end{itemize}

\subsection{Tablas}
\begin{table}[h]
    \centering
    \begin{tabular}{|c|c|c|c|c|}
        \hline
        Palabra1 & Palabra2 & Palabra3 & Palabra4 & Palabra5 \\ \hline
        Albahaca & Brújula  & Cacerola & Delfín   & Estrella \\ \hline
        Faro     & Gorrión  & Hormiga  & Isla     & Jazmín   \\ \hline
        Koala    & Linterna & Marea    & Nieve    & Olivo    \\ \hline
    \end{tabular}
    \caption{Tabla de palabras aleatorias}
    \label{tab:palabras_aleatorias}
\end{table}

\end{document}
