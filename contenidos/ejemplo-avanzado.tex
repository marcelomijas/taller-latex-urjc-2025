% Esta plantilla ha sido creada por Marcelo Marcelo Moreno Porras,
% como parte del taller de Creación de Hojas de Referencia con LaTex
% de las Cuartas Jornadas de Cultura Libre de la Universidad Rey Juan Carlos
% CC-BY-4.0 license

\documentclass{article}

% ----- packages -----
\usepackage{amsmath} % AMS mathematical facilities for LaTeX
\usepackage{enumitem} % Control layout of itemize, enumerate, description
\usepackage{geometry} % Flexible and complete interface to document dimensions
\usepackage{graphicx} % Enhanced support for graphics
\usepackage{xcolor} % Driver-independent color extensions for LaTex and pdfLaTeX

% ----- page customization -----
\geometry{left=3cm, right=1cm, top=2cm, bottom=2cm}

% ----- document -----
\begin{document}

\section{Ejemplos avanzados}
Este es un ejemplo de un documento de \LaTeX que utiliza funciones de paquetes para crear ecuaciones que requieren más complejidad, cambiar los márgenes de la página, realizar listas enumeradas, incluir imágenes o cambiar de color el texto.

\subsection{Ecuaciones más complejas}
La fórmula de Mínimos Cuadrados Ordinarios es:
\begin{equation}
    y = X\beta + u,
\end{equation}
donde:
\[
    y =
    \begin{bmatrix}
        y_1 \\
        y_2 \\
        \vdots \\
        y_n
    \end{bmatrix}, \quad
    X =
    \begin{bmatrix}
        1 & x_{11} & x_{12} & \dots & x_{1k} \\
        1 & x_{21} & x_{22} & \dots & x_{2k} \\
        \vdots & \vdots & \vdots & \ddots & \vdots \\
        1 & x_{n1} & x_{n2} & \dots & x_{nk}
    \end{bmatrix}, \quad
    \beta =
    \begin{bmatrix}
        \beta_0 \\
        \beta_1 \\
        \vdots \\
        \beta_k
    \end{bmatrix}, \quad
    u =
    \begin{bmatrix}
        u_1 \\
        u_2 \\
        \vdots \\
        u_n
    \end{bmatrix}.
\]

\subsection{Listas enumeradas}
Ejemplo de una lista enumerada:
\begin{enumerate}
    \item Primer elemento de la lista
    \item Segundo elemento de la lista
    \begin{itemize}
        \item Item anidado
    \end{itemize}
    \item Tercer elemento de la lista
\end{enumerate}

\subsection{Imágenes}
\begin{figure}[h]
    \centering
    \includegraphics[width=0.3\textwidth]{URJC-Logo.png}
    \caption{Esta es una imagen de muestra.}
    \label{fig:example}
\end{figure}

\subsection{Texto con colores}

Aquí se muestra cómo usar colores en el texto, por ejemplo colorear el texto de \textcolor{red}{color rojo}, o resaltar el texto con \colorbox{yellow}{color amarillo}.

\end{document}

