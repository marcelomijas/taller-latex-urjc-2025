% Esta plantilla ha sido creada por Marcelo Marcelo Moreno Porras,
% como parte del taller de Creación de Hojas de Referencia con LaTex
% de las Cuartas Jornadas de Cultura Libre de la Universidad Rey Juan Carlos
% CC-BY-4.0 license

% ¡Requiere de cambio de compilador a LuaLaTeX!
% Menu -> Compiler -> LuaLaTeX

\documentclass{article}

% ----- packages -----
\usepackage{luacode} % Helper for executing lua code from within TeX

% ----- document -----
\begin{document}

\section{Ejemplo de LuaLaTeX}

Este documento usa \textbf{LuaLaTeX}, lo que nos permite ejecutar código en \textbf{Lua}.

\subsection{Ejemplo de código Lua}
Podemos ejecutar Lua dentro de LaTeX. Por ejemplo, la suma de dos números:

\begin{luacode}
a = 10
b = 20
tex.print("La suma de " .. a .. " y " .. b .. " es " .. (a + b) .. ".")
\end{luacode}

\subsection{Otro ejemplo con Lua}
Podemos definir funciones en Lua y usarlas dentro del documento:

\begin{luacode*}
function cuadrado(x)
  return x * x
end
tex.print("El cuadrado de 5 es " .. cuadrado(5) .. ".")
\end{luacode*}

\end{document}
